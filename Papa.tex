\documentclass[man]{apa6}
\usepackage{lmodern}
\usepackage{amssymb,amsmath}
\usepackage{ifxetex,ifluatex}
\usepackage{fixltx2e} % provides \textsubscript
\ifnum 0\ifxetex 1\fi\ifluatex 1\fi=0 % if pdftex
  \usepackage[T1]{fontenc}
  \usepackage[utf8]{inputenc}
\else % if luatex or xelatex
  \ifxetex
    \usepackage{mathspec}
  \else
    \usepackage{fontspec}
  \fi
  \defaultfontfeatures{Ligatures=TeX,Scale=MatchLowercase}
\fi
% use upquote if available, for straight quotes in verbatim environments
\IfFileExists{upquote.sty}{\usepackage{upquote}}{}
% use microtype if available
\IfFileExists{microtype.sty}{%
\usepackage{microtype}
\UseMicrotypeSet[protrusion]{basicmath} % disable protrusion for tt fonts
}{}
\usepackage{hyperref}
\hypersetup{unicode=true,
            pdftitle={APA Midterm},
            pdfauthor={Angelina Vasquez~\&},
            pdfkeywords={music,social cognition,memory,infant development,open data},
            pdfborder={0 0 0},
            breaklinks=true}
\urlstyle{same}  % don't use monospace font for urls
\usepackage{graphicx,grffile}
\makeatletter
\def\maxwidth{\ifdim\Gin@nat@width>\linewidth\linewidth\else\Gin@nat@width\fi}
\def\maxheight{\ifdim\Gin@nat@height>\textheight\textheight\else\Gin@nat@height\fi}
\makeatother
% Scale images if necessary, so that they will not overflow the page
% margins by default, and it is still possible to overwrite the defaults
% using explicit options in \includegraphics[width, height, ...]{}
\setkeys{Gin}{width=\maxwidth,height=\maxheight,keepaspectratio}
\IfFileExists{parskip.sty}{%
\usepackage{parskip}
}{% else
\setlength{\parindent}{0pt}
\setlength{\parskip}{6pt plus 2pt minus 1pt}
}
\setlength{\emergencystretch}{3em}  % prevent overfull lines
\providecommand{\tightlist}{%
  \setlength{\itemsep}{0pt}\setlength{\parskip}{0pt}}
\setcounter{secnumdepth}{0}
% Redefines (sub)paragraphs to behave more like sections
\ifx\paragraph\undefined\else
\let\oldparagraph\paragraph
\renewcommand{\paragraph}[1]{\oldparagraph{#1}\mbox{}}
\fi
\ifx\subparagraph\undefined\else
\let\oldsubparagraph\subparagraph
\renewcommand{\subparagraph}[1]{\oldsubparagraph{#1}\mbox{}}
\fi

%%% Use protect on footnotes to avoid problems with footnotes in titles
\let\rmarkdownfootnote\footnote%
\def\footnote{\protect\rmarkdownfootnote}


  \title{APA Midterm}
    \author{Angelina Vasquez\textsuperscript{1}~\& \textsuperscript{1,2}}
    \date{}
  
\shorttitle{Music Memory}
\affiliation{
\vspace{0.5cm}
\textsuperscript{1} Brooklyn College\\\textsuperscript{2} }
\keywords{music,social cognition,memory,infant development,open data\newline\indent Word count: X}
\usepackage{csquotes}
\usepackage{upgreek}
\captionsetup{font=singlespacing,justification=justified}

\usepackage{longtable}
\usepackage{lscape}
\usepackage{multirow}
\usepackage{tabularx}
\usepackage[flushleft]{threeparttable}
\usepackage{threeparttablex}

\newenvironment{lltable}{\begin{landscape}\begin{center}\begin{ThreePartTable}}{\end{ThreePartTable}\end{center}\end{landscape}}

\makeatletter
\newcommand\LastLTentrywidth{1em}
\newlength\longtablewidth
\setlength{\longtablewidth}{1in}
\newcommand{\getlongtablewidth}{\begingroup \ifcsname LT@\roman{LT@tables}\endcsname \global\longtablewidth=0pt \renewcommand{\LT@entry}[2]{\global\advance\longtablewidth by ##2\relax\gdef\LastLTentrywidth{##2}}\@nameuse{LT@\roman{LT@tables}} \fi \endgroup}


\DeclareDelayedFloatFlavor{ThreePartTable}{table}
\DeclareDelayedFloatFlavor{lltable}{table}
\DeclareDelayedFloatFlavor*{longtable}{table}
\makeatletter
\renewcommand{\efloat@iwrite}[1]{\immediate\expandafter\protected@write\csname efloat@post#1\endcsname{}}
\makeatother
\usepackage{lineno}

\linenumbers

\authornote{A Brooklyn College Graduate student. Also, a John
Jay College research assistant.

Enter author note here.

Correspondence concerning this article should be addressed to Angelina
Vasquez, Postal address. E-mail:
\href{mailto:my@email.com}{\nolinkurl{my@email.com}}}

\abstract{
Five-month old infants listened to songs sung by their parent, a toy,or
someone unfamiliar for one to two week period. These songs had the same
lyrics and rhythms. However, the melodies were different. The
researchers tested the infants selected attention when a random person
sang the song they were familiar with and the song they were not. The
results indicated that infants paid more attention to the song they were
familiar with, and that exposure time predicted preference time. This
suggests that melodies may carry social meanings for infants.


}

\begin{document}
\maketitle

\section{Methods}\label{methods}

We report how we determined our sample size, all data exclusions (if
any), all manipulations, and all measures in the study.

\subsection{Participants}\label{participants}

The participants were 32 infants and both their parents. However, only
one of the paretns was active in their participation.

\subsection{Material}\label{material}

The materials for this study consisted of two adpated versions of
lullabies from folk collections.These were also provided through a
website with the recordings of the songs and printed versions of the
lyrics. The advanced measures of music audiation assesment was also
used.

\subsection{Procedure}\label{procedure}

This study was compromised of 4 overall experiments. During the first
lab visit for experiment 1, parents were randomly assigned to either
music condition. Once they got their condition with the song they would
have to sing, they were given a music lesson. Parents were also
instructed to visit a website that had the recording of the song so they
could practice. Lastly, parents completed an standard assesment for
music perception skill.

Everyday, participants were emailed a survey to assess the number of
times the infant heard the song. The researchers took the average of
those surveys and mulitpled it by the days of the study to determine how
much exposure they had to that song.

On the last day of the study, the infants were given an attention test.
They sat on their parent's lap while they watched a projection screen
with another two people on it singing the familar song and unfamilar
song.The first were presented with vidoe recordings of the individuals
smiling. Then they viewed the recordings with the individuals singing.
Lastly, the viewed the the first recording again of the smiling
individuals.

In experiment two, the overal procedure was replicated fro, experiment
1. However, instead of having the parent sing the song to them duriing
the study, they had a toy with a reocding of the song sing it during the
study.

In experiment three, the overall procedure was replicated fro,
experiment 1. However, instead of having the parent sing the song to
them duriing the study, they had a friendly adult.

\subsection{Data analysis}\label{data-analysis}

We used R (Version 3.5.2; R Core Team, 2018) and the R-packages
\emph{data.table} (Version 1.12.0; Dowle \& Srinivasan, 2019),
\emph{dplyr} (Version 0.8.0.1; Wickham, François, Henry, \& Müller,
2019), \emph{papaja} (Version 0.1.0.9842; Aust \& Barth, 2018), and
\emph{summarytools} (Version 0.9.2; Comtois, 2019) for all our analyses.

\begin{verbatim}
## [1] 0.5210967
\end{verbatim}

\begin{verbatim}
## [1] 0.1769651
\end{verbatim}

\includegraphics{Papa_files/figure-latex/unnamed-chunk-1-1.pdf} T-test
analysis

\begin{verbatim}
## 
##  One Sample t-test
## 
## data:  baseline
## t = 0.67438, df = 31, p-value = 0.5051
## alternative hypothesis: true mean is not equal to 0.5
## 95 percent confidence interval:
##  0.4572940 0.5848994
## sample estimates:
## mean of x 
## 0.5210967
\end{verbatim}

So, there we have it. We did a one-sample t-test. Here's how you would
report it, t(31) = .67, p = .505. Or, we might say something like:

During the baseline condition, the mean proportion looking time toward
the singer was .52, and was not significantly different from .5,
according to a one-sample test, t(31) = .67, p = .505.

\subsubsection{power anaylisis probablity finding something given it is
there how big is it how many subjects there is the pwr package to do
analysis go to github simulation presentations for power
analysis}\label{power-anaylisis-probablity-finding-something-given-it-is-there-how-big-is-it-how-many-subjects-there-is-the-pwr-package-to-do-analysis-go-to-github-simulation-presentations-for-power-analysis}

\section{Results}\label{results}

\subsubsection{apa print function will make a table of the data you
ran(anova table) if you write something and then put`r write something
other tick-that treats as r code some this test significant tick r F
value p value tick-----a=1,2,3,sapply(a,fun=function(x)return
(x+1))}\label{apa-print-function-will-make-a-table-of-the-data-you-rananova-table-if-you-write-something-and-then-putr-write-something-other-tick-that-treats-as-r-code-some-this-test-significant-tick-r-f-value-p-value-ticka123sapplyafunfunctionxreturn-x1}

\begin{figure}
\centering
\includegraphics{Papa_files/figure-latex/myfig-1.pdf}
\caption{\label{fig:myfig}This is histo}
\end{figure}

\section{Discussion}\label{discussion}

\newpage

\section{References}\label{references}

Mehr, S. A., Song, L. A., \& Spelke, E. S. (2016). For 5-month-old
infants, melodies are social. Psychological Science, 27(4), 486-501.

\begingroup
\setlength{\parindent}{-0.5in} \setlength{\leftskip}{0.5in}

\hypertarget{refs}{}
\hypertarget{ref-R-papaja}{}
Aust, F., \& Barth, M. (2018). \emph{papaja: Create APA manuscripts with
R Markdown}. Retrieved from \url{https://github.com/crsh/papaja}

\hypertarget{ref-R-summarytools}{}
Comtois, D. (2019). \emph{Summarytools: Tools to quickly and neatly
summarize data}. Retrieved from
\url{https://CRAN.R-project.org/package=summarytools}

\hypertarget{ref-R-data.table}{}
Dowle, M., \& Srinivasan, A. (2019). \emph{Data.table: Extension of
`data.frame`}. Retrieved from
\url{https://CRAN.R-project.org/package=data.table}

\hypertarget{ref-R-base}{}
R Core Team. (2018). \emph{R: A language and environment for statistical
computing}. Vienna, Austria: R Foundation for Statistical Computing.
Retrieved from \url{https://www.R-project.org/}

\hypertarget{ref-R-dplyr}{}
Wickham, H., François, R., Henry, L., \& Müller, K. (2019). \emph{Dplyr:
A grammar of data manipulation}. Retrieved from
\url{https://CRAN.R-project.org/package=dplyr}

\endgroup


\end{document}
