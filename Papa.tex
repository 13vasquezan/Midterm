\documentclass[man]{apa6}
\usepackage{lmodern}
\usepackage{amssymb,amsmath}
\usepackage{ifxetex,ifluatex}
\usepackage{fixltx2e} % provides \textsubscript
\ifnum 0\ifxetex 1\fi\ifluatex 1\fi=0 % if pdftex
  \usepackage[T1]{fontenc}
  \usepackage[utf8]{inputenc}
\else % if luatex or xelatex
  \ifxetex
    \usepackage{mathspec}
  \else
    \usepackage{fontspec}
  \fi
  \defaultfontfeatures{Ligatures=TeX,Scale=MatchLowercase}
\fi
% use upquote if available, for straight quotes in verbatim environments
\IfFileExists{upquote.sty}{\usepackage{upquote}}{}
% use microtype if available
\IfFileExists{microtype.sty}{%
\usepackage{microtype}
\UseMicrotypeSet[protrusion]{basicmath} % disable protrusion for tt fonts
}{}
\usepackage{hyperref}
\hypersetup{unicode=true,
            pdftitle={APA Midterm},
            pdfauthor={Angelina Vasquez~\&},
            pdfkeywords={music,social cognition,memory,infant development,open data},
            pdfborder={0 0 0},
            breaklinks=true}
\urlstyle{same}  % don't use monospace font for urls
\usepackage{graphicx,grffile}
\makeatletter
\def\maxwidth{\ifdim\Gin@nat@width>\linewidth\linewidth\else\Gin@nat@width\fi}
\def\maxheight{\ifdim\Gin@nat@height>\textheight\textheight\else\Gin@nat@height\fi}
\makeatother
% Scale images if necessary, so that they will not overflow the page
% margins by default, and it is still possible to overwrite the defaults
% using explicit options in \includegraphics[width, height, ...]{}
\setkeys{Gin}{width=\maxwidth,height=\maxheight,keepaspectratio}
\IfFileExists{parskip.sty}{%
\usepackage{parskip}
}{% else
\setlength{\parindent}{0pt}
\setlength{\parskip}{6pt plus 2pt minus 1pt}
}
\setlength{\emergencystretch}{3em}  % prevent overfull lines
\providecommand{\tightlist}{%
  \setlength{\itemsep}{0pt}\setlength{\parskip}{0pt}}
\setcounter{secnumdepth}{0}
% Redefines (sub)paragraphs to behave more like sections
\ifx\paragraph\undefined\else
\let\oldparagraph\paragraph
\renewcommand{\paragraph}[1]{\oldparagraph{#1}\mbox{}}
\fi
\ifx\subparagraph\undefined\else
\let\oldsubparagraph\subparagraph
\renewcommand{\subparagraph}[1]{\oldsubparagraph{#1}\mbox{}}
\fi

%%% Use protect on footnotes to avoid problems with footnotes in titles
\let\rmarkdownfootnote\footnote%
\def\footnote{\protect\rmarkdownfootnote}


  \title{APA Midterm}
    \author{Angelina Vasquez\textsuperscript{1}~\& \textsuperscript{1,2}}
    \date{}
  
\shorttitle{Title}
\affiliation{
\vspace{0.5cm}
\textsuperscript{1} Brooklyn College\\\textsuperscript{2} }
\keywords{music,social cognition,memory,infant development,open data\newline\indent Word count: X}
\usepackage{csquotes}
\usepackage{upgreek}
\captionsetup{font=singlespacing,justification=justified}

\usepackage{longtable}
\usepackage{lscape}
\usepackage{multirow}
\usepackage{tabularx}
\usepackage[flushleft]{threeparttable}
\usepackage{threeparttablex}

\newenvironment{lltable}{\begin{landscape}\begin{center}\begin{ThreePartTable}}{\end{ThreePartTable}\end{center}\end{landscape}}

\makeatletter
\newcommand\LastLTentrywidth{1em}
\newlength\longtablewidth
\setlength{\longtablewidth}{1in}
\newcommand{\getlongtablewidth}{\begingroup \ifcsname LT@\roman{LT@tables}\endcsname \global\longtablewidth=0pt \renewcommand{\LT@entry}[2]{\global\advance\longtablewidth by ##2\relax\gdef\LastLTentrywidth{##2}}\@nameuse{LT@\roman{LT@tables}} \fi \endgroup}


\DeclareDelayedFloatFlavor{ThreePartTable}{table}
\DeclareDelayedFloatFlavor{lltable}{table}
\DeclareDelayedFloatFlavor*{longtable}{table}
\makeatletter
\renewcommand{\efloat@iwrite}[1]{\immediate\expandafter\protected@write\csname efloat@post#1\endcsname{}}
\makeatother
\usepackage{lineno}

\linenumbers

\authornote{A Brooklyn College Graduate student. Also, a John
Jay College research assistant.

Enter author note here.

Correspondence concerning this article should be addressed to Angelina
Vasquez, Postal address. E-mail:
\href{mailto:my@email.com}{\nolinkurl{my@email.com}}}

\abstract{
Five-month old infants listened to songs sung by their parent, a toy,or
someone unfamiliar for one to two week period. These songs had the same
lyrics and rhythms. However, the melodies were different. The
researchers tested the infants selected attention when a random person
sang the song they were familiar with and the song they were not. The
results indicated that infants paid more attention to the song they were
familiar with, and that exposure time predicted preference time. This
suggests that melodies may carry social meanings for infants.

Two to three sentences of \textbf{more detailed background},
comprehensible to scientists in related disciplines.

One sentence clearly stating the \textbf{general problem} being
addressed by this particular study.

One sentence summarizing the main result (with the words ``\textbf{here
we show}'' or their equivalent).

Two or three sentences explaining what the \textbf{main result} reveals
in direct comparison to what was thought to be the case previously, or
how the main result adds to previous knowledge.

One or two sentences to put the results into a more \textbf{general
context}.

Two or three sentences to provide a \textbf{broader perspective},
readily comprehensible to a scientist in any discipline.


}

\begin{document}
\maketitle

\section{Methods}\label{methods}

We report how we determined our sample size, all data exclusions (if
any), all manipulations, and all measures in the study.

\subsection{Participants}\label{participants}

\subsection{Material}\label{material}

\subsection{Procedure}\label{procedure}

\subsection{Data analysis}\label{data-analysis}

We used R (Version 3.5.2; R Core Team, 2018) and the R-packages
\emph{data.table} (Version 1.12.0; Dowle \& Srinivasan, 2019),
\emph{dplyr} (Version 0.8.0.1; Wickham, François, Henry, \& Müller,
2019), \emph{papaja} (Version 0.1.0.9842; Aust \& Barth, 2018), and
\emph{summarytools} (Version 0.9.2; Comtois, 2019) for all our analyses.

\begin{verbatim}
## [1] 0.5210967
\end{verbatim}

\begin{verbatim}
## [1] 0.1769651
\end{verbatim}

\includegraphics{Papa_files/figure-latex/unnamed-chunk-1-1.pdf} T-test
analysis

\begin{verbatim}
## 
##  One Sample t-test
## 
## data:  baseline
## t = 0.67438, df = 31, p-value = 0.5051
## alternative hypothesis: true mean is not equal to 0.5
## 95 percent confidence interval:
##  0.4572940 0.5848994
## sample estimates:
## mean of x 
## 0.5210967
\end{verbatim}

So, there we have it. We did a one-sample t-test. Here's how you would
report it, t(31) = .67, p = .505. Or, we might say something like:

During the baseline condition, the mean proportion looking time toward
the singer was .52, and was not significantly different from .5,
according to a one-sample test, t(31) = .67, p = .505.

\subsubsection{power anaylisis probablity finding something given it is
there how big is it how many subjects there is the pwr package to do
analysis go to github simulation presentations for power
analysis}\label{power-anaylisis-probablity-finding-something-given-it-is-there-how-big-is-it-how-many-subjects-there-is-the-pwr-package-to-do-analysis-go-to-github-simulation-presentations-for-power-analysis}

\section{Results}\label{results}

\subsubsection{apa print function will make a table of the data you
ran(anova table) if you write something and then put`r write something
other tick-that treats as r code some this test significant tick r F
value p value tick-----a=1,2,3,sapply(a,fun=function(x)return
(x+1))}\label{apa-print-function-will-make-a-table-of-the-data-you-rananova-table-if-you-write-something-and-then-putr-write-something-other-tick-that-treats-as-r-code-some-this-test-significant-tick-r-f-value-p-value-ticka123sapplyafunfunctionxreturn-x1}

\begin{figure}
\centering
\includegraphics{Papa_files/figure-latex/myfig-1.pdf}
\caption{\label{fig:myfig}This is histo}
\end{figure}

\section{Discussion}\label{discussion}

\newpage

\section{References}\label{references}

Mehr, S. A., Song, L. A., \& Spelke, E. S. (2016). For 5-month-old
infants, melodies are social. Psychological Science, 27(4), 486-501.

\begingroup
\setlength{\parindent}{-0.5in} \setlength{\leftskip}{0.5in}

\hypertarget{refs}{}
\hypertarget{ref-R-papaja}{}
Aust, F., \& Barth, M. (2018). \emph{papaja: Create APA manuscripts with
R Markdown}. Retrieved from \url{https://github.com/crsh/papaja}

\hypertarget{ref-R-summarytools}{}
Comtois, D. (2019). \emph{Summarytools: Tools to quickly and neatly
summarize data}. Retrieved from
\url{https://CRAN.R-project.org/package=summarytools}

\hypertarget{ref-R-data.table}{}
Dowle, M., \& Srinivasan, A. (2019). \emph{Data.table: Extension of
`data.frame`}. Retrieved from
\url{https://CRAN.R-project.org/package=data.table}

\hypertarget{ref-R-base}{}
R Core Team. (2018). \emph{R: A language and environment for statistical
computing}. Vienna, Austria: R Foundation for Statistical Computing.
Retrieved from \url{https://www.R-project.org/}

\hypertarget{ref-R-dplyr}{}
Wickham, H., François, R., Henry, L., \& Müller, K. (2019). \emph{Dplyr:
A grammar of data manipulation}. Retrieved from
\url{https://CRAN.R-project.org/package=dplyr}

\endgroup


\end{document}
